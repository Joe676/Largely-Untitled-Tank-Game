\chapter{Wst�p}
\section{Cel pracy}

Celem niniejszej pracy dyplomowej jest zaprojektowanie oraz zaprogramowanie sieciowej gry wieloosobowej.

Gra b�dzie umo�liwia� po��czenie od dw�ch do sze�ciu os�b w sieci lokalnej. Aplikacja zostanie przygotowana na komputery PC z systemem Windows 10. 
Serwerem gry b�dzie komputer jednego z graczy, nazywanego r�wnie� \emph{hostem}.
Zar�wno mechaniki gry jak i system sieciowy zostan� zaimplementowane z wykorzystaniem silnika Godot. (cite: godot page)
W celu zapewnienia p�ynnej i komfortowej rozgrywki zostan� zastosowane techniki predykcji klienta.

Pocz�tkowo przygotowane zostan�:
\begin{itemize}
    \item \emph{Koncepcja gry} - projekt mechanik gry, projekt wizualny, elementy \emph{game designu};
    \item \emph{Projekt aplikacji} - projekt interfejsu u�ytkownika, plan przygotowania systemu sieciowego oraz dzia�anie mechanik;
    \item \emph{Prototyp aplikacji} - pocz�tkowa, okrojona wersja gry, przygotowana w celu zapoznania si� z dzia�aniem silnika oraz w celu okre�lenia potencjalnych problem�w z p�niejsz� implementacj�. 
\end{itemize}

Powy�sze elementy b�d� przygotowywane r�wnolegle, aby wiedza pozyskana w czasie rozwijania jednego mog�a mo�liwie najlepiej wesprze� pozosta�e.

\section{Technologie}
W rozwoju projektu wykorzystane zostan� poni�sze technologie:
\begin{itemize}
    \item \emph{Godot} - Silnik gier; implementacja mechanik gry, silnik fizyczny, komunikacja sieciowa, wysokopoziomowy interfejs sieciowy;
    \item \emph{Git}, \emph{GitHub} - System kontroli wersji;
    \item \emph{Blender} - Modelowanie 3D, przygotowanie materia��w;
    \item \emph{Trello} - Planowanie, harmonogram pracy;
\end{itemize}