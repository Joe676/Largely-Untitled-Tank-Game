\abstract{
%Stworzenie usunąć
Celem niniejszej pracy dyplomowej było zaprojektowanie i stworzenie wieloosobowej gry sieciowej. Gra miała zostać stworzona z wykorzystaniem silnika Godot. Gra wykorzystuje grafikę 3D oraz widok trzecioosobowy. Aplikacja umożliwia połączenie od dwóch do sześciu graczy w sieci lokalnej LAN. Stworzona aplikacja została przetestowana jednostkowo i integracyjnie z wykorzystaniem narzędzia GUT (\emph{Godot Unit Testing}) oraz manualnie pod systemem operacyjnym Windows 10.  

Przed rozpoczęciem implementacji przygotowana została koncepcja gry, plan mechanik oraz najważniejszych systemów. W następnej kolejności stworzony został prototyp. Przygotowując prototyp przeprowadzona została analiza trudności implementacji systemu sieciowego. 

Zaprojektowane zostały interfejsy użytkownika oraz najważniejsze podsystemy. W następnej kolejności zostały one zaimplementowane zgodnie z koncepcją i projektami. Została również przeprowadzona dyskusja wyników, analiza wniosków. Opisano również możliwości dalszego rozwoju projektu.
}{
The goal of this thesis was to design and create a game featuring an on-line multiplayer using the Godot game engine. The game uses 3D graphics, providing player with third person view. Application enables two to six players to connect over local network. Unit and integration tests were created using the GUT toolkit. Manual tests were also ran under Windows 10.

Before implementing the software, a game design concept was formulated. It consists of plans for mechanics and systems to be featured in the game. A prototype was created in order to analyze the difficulty of implementing the network system. 

Designs included graphical interfaces and the most important subsystems. Those were implemented according to concepts and designs. Finally, the results were analyzed and some ways of improving the project in the future were proposed.
}