\chapter*{Wstęp}
\section*{Cel pracy}

Celem niniejszej pracy dyplomowej jest zaprojektowanie oraz zaprogramowanie sieciowej gry wieloosobowej.

Gra będzie umożliwiać połączenie od dwóch do sześciu osób w sieci lokalnej. Aplikacja zostanie przygotowana na komputery PC z systemem Windows 10. Gra będzie przygotowana z myślą o sterowaniu klawiaturą i myszą.
Serwerem gry będzie komputer jednego z graczy, nazywanego również \emph{hostem}.
Zarówno mechaniki gry jak i system sieciowy zostaną zaimplementowane z wykorzystaniem silnika Godot. (cite: godot page)
W celu zapewnienia płynnej i komfortowej rozgrywki zostaną zastosowane techniki predykcji klienta.
Oprogramowanie i interfejs użytkownika zostaną przygotowane w języku angielskim.

Początkowo przygotowane zostaną:
\begin{itemize}
    \item \emph{Koncepcja gry} - projekt mechanik gry, projekt wizualny, elementy \emph{game designu};
    \item \emph{Projekt aplikacji} - projekt interfejsu użytkownika, plan przygotowania systemu sieciowego oraz działanie mechanik;
    \item \emph{Prototyp aplikacji} - początkowa, okrojona wersja gry, przygotowana w celu zapoznania się z działaniem silnika oraz w celu określenia potencjalnych problemów z późniejszą implementacją. 
\end{itemize}

Powyższe elementy będą przygotowywane równolegle, aby wiedza pozyskana w czasie rozwijania jednego mogła możliwie najlepiej wesprzeć pozostałe.

\section*{Technologie}
W rozwoju projektu wykorzystane zostaną poniższe technologie:
\begin{itemize}
    \item \emph{Godot} - Silnik gier; implementacja mechanik gry, silnik fizyczny, komunikacja sieciowa, wysokopoziomowy interfejs sieciowy;
    \item \emph{Git}, \emph{GitHub} - System kontroli wersji;
    \item \emph{Blender} - Modelowanie 3D, przygotowanie materiałów;
    \item \emph{Trello} - Planowanie, harmonogram pracy;
\end{itemize}