\chapter{Analiza rynku}
\section{Podobne rozwiązania}
% Na rynku dostępnych jest wiele gier przypominających w pewnym stopniu tworzony projekt. 
% Poniżej opisane zostały popularne gry, którymi inspirowany jest tworzony projekt:

\subsection{ROUNDS}
,,ROUNDS'' jest konkurencyjną grą dwuosobową wydaną w 2021 roku przez studio Landfall Games\cite{rounds}. Umożliwia grę lokalnie, na jednym urządzeniu lub przez sieć korzystając z serwerów Steam. Rozgrywka prowadzona jest w rundach, między którymi gracz przegrywający otrzymuje karty wzmacniające jego możliwości. W grze zawartych jest ponad 60 kart modyfikujących różne atrybuty postaci. Rozgrywka odbywa się w płaszczyźnie dwuwymiarowej z widokiem ,,z boku''. Oznacza to, że zarówno na postaci graczy jak i na pociski oddziałuje grawitacja. Dodaje to znaczną głębię rozgrywki. Ponadto oprócz poruszania i strzału gracz ma do wykorzystania kolejną interakcję - blok. Blokując gracz przez krótką chwilę może odbić nadlatujące pociski. Dodatkowo wiele kart dodaje dodatkowe zdolności do bloku zwiększające jego możliwości bojowe.

\subsection{Boomerang Fu}
Wydane w 2020 roku przez studio Cranky Watermelon ,,Boomerang Foo'' jest imprezową grą akcji\cite{boomerang_fu}. Umożliwia grę wieloosobową dla 2 do 6 graczy na jednym urządzeniu. Gracze sterują postaciami, korzystając z bumerangów w celu wzajemnej eliminacji. Po każdej z rund gracze dostają punkty w zależności od liczby wyeliminowanych przeciwników. W czasie rozgrywki gracze mają możliwość otrzymania wzmocnień poprzez zebranie ich na planszy. Gracz może jednocześnie mieć maksimum trzy wzmocnienia. Zmieniają one zachowanie bumerangu lub postaci i nie modyfikują żadnych charakterystyk na stałe.

\subsection{World of Tanks}
Wydana przez studio Wargaming.net sieciowa gra ,,World of Tanks'' jest sieciową strzelanką, w której gracze sterują czołgami\cite{worldoftanks}. Gra udostępnia ponad 600 różnych modeli czołgów z okresu 20. wieku. W związku ze skupieniem na historycznych modelach czołgów gra celuje w realistyczne oddanie wyglądu i mechaniki działania maszyn bojowych. Ponadto realizm osiągany jest również w samym świecie gry. Na wielu dostępnych poziomach istotne są nie tylko przeszkody fizyczne ale również ukształtowanie terenu ograniczające lub zwiększające możliwości gracza. Gra udostępnia też wiele różnorodnych trybów rozgrywki.


\subsection{Wii Play: Tanks!}
,,Tanks!'' jest jedną z minigier zawartych w zestawie ,,Wii Play'' dostępnym na konsolę Nintendo Wii\cite{wii_tanks}. Jest to gra jednoosobowa w której gracz steruje czołgiem próbując wyeliminować sterowane przez sztuczną inteligencję czołgi przeciwników. Każdy z poziomów składa się z ustalonego zestawu wrogów. Gra prowadzona jest z trzeciej osoby ze stacjonarnyum widokiem izometrycznym. Dostępny jest również lokalny wieloosobowy tryb częściowo-kooperacyjny. W tym trybie dwóch graczy rozgrywa poziomy jednoosobowe wspólnie, zdobywając punkty za każdego wyeliminowanego przeciwnika. Czołgi przeciwników należą do jednego z kilku rodzajów, każdy rodzaj posiada inne charakterystyki poruszania i strzałów.   
